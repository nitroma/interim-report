% !TeX root = ../main.tex
\section{Decision Analysis}
\label{app:matrix}

\subsection{Sample AHP and TOPSIS calculation for product selection}

1. Pair-wise comparison of criteria: the scale developed by Saaty was adopted to assess the relative importance of one criteria to another. The fundamental scale is as follows: 1 for equal importance, 3 for moderate importance of one criteria over the other, 5 for essential or strong importance, 7 for very strong importance and 9 for extreme importance. Even numbers can be used as intermediate values. The normalised weightings were then calculated . he results for product selection are shown in Table


2. Validation of consistency: The consistency between the weightings was verified with a method proposed by Saaty [cite]. The largest eigenvalue ($\lambda_{max}$) was determine with the following formula where $i$ refers to the criteria:
\begin{equation}
    \lambda_{max}=\sum^{i} \mathrm{w}_{i}\cdot \mathrm{Sum}(i)
\end{equation}

The consistency index (CI) is calculated for the number of criteria $n$:
\begin{equation}
   \mathrm{CI} = \frac{\lambda_{max}-n}{n-1}
\end{equation}

The consistency ratio CI/RI is then calculated using the Random Consistency Index (RI) retrieved from (Saaty, 1980). 

For product selection, $\lambda_{max}$=3.1218, CI=0.0409 and RI=0.58, hence the consistency ratio is worth xx. Weightings are deemed consistent if the ratio is less than 10\%.\\

3. AHP score calculation:

4. Weighted Normalised Matrix for TOPSIS:

5. Euclidean distance from ideal best and ideal worst calculation:

6. Evaluation of Performance Score and ranking:

\subsection{Summary of production routes}

\subsection{Plant location}

\subsection{Nitration catalyst selection}

\subsection{Solvent for 2-nitrotoluene hydrogenation}

