% !TeX root = ../main.tex
\section{Unit Sizing Methodology}
\label{app:sizing}
\subsection{Reactor Sizing}
%Based on X paper, (include why reactors are PFR) (reference Levenspiel)
According to Froment/Bischoff, all 4 reactors chosen (packed-bed microreactor, trickle bed reactor, xxx, xxx) can be assumed to have the same [froment bischoff][tricklebed], which can be defined in the equation below:

The packed-bed microreactor and trickle bed reactor are assumed to have the same design equation as a plug flow reactor, as shown below:

\begin{equation}
    V_R = \int_{F_{A0}}^{F_{A}} \frac{dF_A}{-r_A} = \int_{F_{A0}}^{F_{A}} \frac{dF_A}{-kC_A}
    \label{reactor_sizing}
\end{equation}

Packed bed reactors and microreactors are commonly assumed to model plug flow behaviour 

\paragraph{}{Key assumption}
 
\indent 1.  Using the rate equation in the design equation accounts for intrinsic kinetics but not for mass transfer limitations when sizing the reactor. The effects of mass transfer limitations on the rate might be significant for the hydrogenation of ONT to o-toluidine as the rate is dependent on the catalyst. Including the effectiveness factor in the rate equation will provide a more accurate representation of the volume, but these limitations are not considered at this stage.

2.  Catalysts are used in all of the reactors in the design. For more accurate sizing of the reactors, the deactivation of the catalyst should be reflected in the rate equation given by the deactivation function. The catalysts used in the reactors have a relatively long deactivation time [], so the deactivation of the catalysts is considered to be negligible at this stage. 

\subsection{Separation Unit Sizing}
\begin{table}[h]
\centering
    \caption{Key assumptions for sizing the separator units with references}
    \label{tab:assumptions-of-sizing-separator-units}\footnotesize
\resizebox{\textwidth}{!}{%
\begin{tabular}{llll}
\hline
Designated name & Type                                                          & Assumptions involved in sizing                                                                                                                                                                                & Source                                                                     \\ \hline
S101            & Decanter                                                      & Aqueous phase is continuous; organic phase is dispersed                                                                                                                                                   & Ludwig et al. 1999                                                         \\
S102            & \begin{tabular}[c]{@{}l@{}}Distillation\\ column\end{tabular} & \begin{tabular}[c]{@{}l@{}}Diameter of plate column estimated using methodology proposed in Seider et al. 2009;\\ height to diameter ratio designed to be 25 in accordance with Douglas 1988\end{tabular} & \begin{tabular}[c]{@{}l@{}}Seider et al. 2009;\\ Douglas 1988\end{tabular} \\
S103            & \begin{tabular}[c]{@{}l@{}}Distillation\\ column\end{tabular} & \begin{tabular}[c]{@{}l@{}}Diameter of plate column estimated using methodology proposed in Seider et al. 2009;\\ height to diameter ratio designed to be 25 in accordance with Douglas 1988\end{tabular} & \begin{tabular}[c]{@{}l@{}}Seider et al. 2009;\\ Douglas 1989\end{tabular} \\
S201            & \begin{tabular}[c]{@{}l@{}}Distillation\\ column\end{tabular} & \begin{tabular}[c]{@{}l@{}}Diameter of plate column estimated using methodology proposed in Seider et al. 2009;\\ height to diameter ratio designed to be 25 in accordance with Douglas 1988\end{tabular} & \begin{tabular}[c]{@{}l@{}}Seider et al. 2009;\\ Douglas 1999\end{tabular} \\
S202            & Crystalliser                                                  & Scaling from past example (Sulzer falling-film melt crystalliser)                                                                                                                                         & Seader et al. 2011                                                         \\
S301            & Flash drum                                                    & Minimum residence time of liquid phase is 5 min; vessel half-filled with liquid                                                                                                                           & Seader et al. 2011                                                         \\
S302            & \begin{tabular}[c]{@{}l@{}}Distillation\\ column\end{tabular} & \begin{tabular}[c]{@{}l@{}}Diameter of plate column estimated using methodology proposed in Seider et al. 2009;\\ height to diameter ratio designed to be 25 in accordance with Douglas 1988\end{tabular} & \begin{tabular}[c]{@{}l@{}}Seider et al. 2009;\\ Douglas 1990\end{tabular} \\
S303            & \begin{tabular}[c]{@{}l@{}}Distillation\\ column\end{tabular} & \begin{tabular}[c]{@{}l@{}}Diameter of plate column estimated using methodology proposed in Seider et al. 2009;\\ height to diameter ratio designed to be 25 in accordance with Douglas 1988\end{tabular} & \begin{tabular}[c]{@{}l@{}}Seider et al. 2009;\\ Douglas 1991\end{tabular} \\
S401            & Flash drum                                                    & Minimum residence time of liquid phase is 5 min; vessel half-filled with liquid                                                                                                                           & Seader et al. 2011                                                         \\
S402            & Crystalliser                                                  & Scaling from past example (Sulzer falling-film melt crystalliser)                                                                                                                                         & Seader et al. 2012                                                         \\
S501            & Flash drum                                                    & Minimum residence time of liquid phase is 5 min; vessel half-filled with liquid                                                                                                                           & Seader et al. 2013                                                         \\
S502            & \begin{tabular}[c]{@{}l@{}}Distillation\\ column\end{tabular} & \begin{tabular}[c]{@{}l@{}}Diameter of plate column estimated using methodology proposed in Seider et al. 2009;\\ height to diameter ratio designed to be 25 in accordance with Douglas 1988\end{tabular} & \begin{tabular}[c]{@{}l@{}}Seider et al. 2009;\\ Douglas 1991\end{tabular} \\
S503            & \begin{tabular}[c]{@{}l@{}}Distillation\\ column\end{tabular} & \begin{tabular}[c]{@{}l@{}}Diameter of plate column estimated using methodology proposed in Seider et al. 2009;\\ height to diameter ratio designed to be 25 in accordance with Douglas 1988\end{tabular} & \begin{tabular}[c]{@{}l@{}}Seider et al. 2009;\\ Douglas 1992\end{tabular} \\ \hline
\end{tabular}%
}
\end{table}

%Key assumptions and sources of reference for sizing the separator units are summarised in Table \ref{tab:assumptions of sizing separator units}.
