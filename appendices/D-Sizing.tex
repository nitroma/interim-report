% !TeX root = ../main.tex
\begin{landscape}
\section{Unit Sizing Methodology}
\label{app:sizing}
\subsection{Reactor Sizing}
%Based on X paper, (include why reactors are PFR) (reference Levenspiel)
According to Froment/Bischoff, all 4 reactors chosen (packed-bed microreactor, trickle bed reactor, xxx, xxx) can be assumed to have the same design equation as a plug flow reactor, which can be defined in the equation below:
The packed-bed microreactor and trickle bed reactor are assumed to have the same 

\begin{equation}
    V_R = \int_{F_{A0}}^{F_{A}} \frac{dF_A}{-r_A} = \int_{F_{A0}}^{F_{A}} \frac{dF_A}{-kC_A}
    \label{reactor_sizing}
\end{equation}

Packed bed microreactors are assumed to model plug flow behaviour
Future work will involve more accurate sizing of the reactors
\subsection{Jaksland Procedure for Ranking Separation Techniques}
The Jaksland procedure was chosen as a quantitative method of ranking separation techniques because it is well established and does not require much computation once physical properties of the chemicals are known []. The basic steps are 
\begin{table}[h]
\centering
\caption{Caption}
\begin{tabular}{@{}lllllllllllllll@{}}
\toprule
Ref & Pair              & r\_mr & r\_critT & r\_critP & r\_triple & r\_molvol & r\_vapP & r\_bp & r\_mp & r\_Hvap & r\_Hfus  & r\_vdwvol & r\_kd    & r\_sol   \\ \midrule
1   & 2-nitro / 4-nitro & 1.00  & 1.03     & 1.72     & 1.00      & \#DIV/0!  & 18.10   & 1.03  & 1.21  & 1.03    & \#DIV/0! & \#DIV/0!  & \#DIV/0! & \#DIV/0! \\
2   & 2-nitro / 3-nitro & 1.00  & 1.02     & 1.00     & 1.00      & \#DIV/0!  & 21.49   & 1.02  & 1.07  & 1.01    & \#DIV/0! & 1.00      & \#DIV/0! & \#DIV/0! \\
3   & 4-nitro / 3-nitro & 1.00  & 1.01     & 1.72     & 1.00      & \#DIV/0!  & 1.19    & 1.01  & 1.13  & 1.02    & \#DIV/0! & \#DIV/0!  & \#DIV/0! & \#DIV/0! \\
4   & 2-nitro / tol     & 1.49  & 1.21     & 1.08     & 1.82      & \#DIV/0!  & 3000.51 & 1.29  & 1.51  & 1.59    & \#DIV/0! & \#DIV/0!  & \#DIV/0! & \#DIV/0! \\
5   & 4-nitro / tol     & 1.49  & 1.25     & 1.86     & 1.82      & \#DIV/0!  & 165.80  & 1.33  & 1.82  & 1.63    & 2.36     & \#DIV/0!  & \#DIV/0! & \#DIV/0! \\
6   & 3-nitro / tol     & 1.49  & 1.24     & 1.08     & 1.82      & \#DIV/0!  & 139.63  & 1.31  & 1.62  & 1.61    & \#DIV/0! & \#DIV/0!  & \#DIV/0! & \#DIV/0! \\ \bottomrule
\end{tabular}
\end{table}
\label{app:sizing}


Key assumptions and sources of reference for sizing the separator units are summarised in Table \ref{tab:assumptions of sizing separator units}.


\begin{table}[h]
\centering
    \caption{Key assumptions for sizing the separator units with references}
    \label{tab:assumptions of sizing separator units}\footnotesize
\resizebox{\textwidth}{!}{%
\begin{tabular}{llll}
\hline
Designated name & Type                                                          & Assumptions involved in sizing                                                                                                                                                                                & Source                                                                     \\ \hline
S101            & Decanter                                                      & Aqueous phase is continuous; organic phase is dispersed                                                                                                                                                   & Ludwig et al. 1999                                                         \\
S102            & \begin{tabular}[c]{@{}l@{}}Distillation\\ column\end{tabular} & \begin{tabular}[c]{@{}l@{}}Diameter of plate column estimated using methodology proposed in Seider et al. 2009;\\ height to diameter ratio designed to be 25 in accordance with Douglas 1988\end{tabular} & \begin{tabular}[c]{@{}l@{}}Seider et al. 2009;\\ Douglas 1988\end{tabular} \\
S103            & \begin{tabular}[c]{@{}l@{}}Distillation\\ column\end{tabular} & \begin{tabular}[c]{@{}l@{}}Diameter of plate column estimated using methodology proposed in Seider et al. 2009;\\ height to diameter ratio designed to be 25 in accordance with Douglas 1988\end{tabular} & \begin{tabular}[c]{@{}l@{}}Seider et al. 2009;\\ Douglas 1989\end{tabular} \\
S201            & \begin{tabular}[c]{@{}l@{}}Distillation\\ column\end{tabular} & \begin{tabular}[c]{@{}l@{}}Diameter of plate column estimated using methodology proposed in Seider et al. 2009;\\ height to diameter ratio designed to be 25 in accordance with Douglas 1988\end{tabular} & \begin{tabular}[c]{@{}l@{}}Seider et al. 2009;\\ Douglas 1999\end{tabular} \\
S202            & Crystalliser                                                  & Scaling from past example (Sulzer falling-film melt crystalliser)                                                                                                                                         & Seader et al. 2011                                                         \\
S301            & Flash drum                                                    & Minimum residence time of liquid phase is 5 min; vessel half-filled with liquid                                                                                                                           & Seader et al. 2011                                                         \\
S302            & \begin{tabular}[c]{@{}l@{}}Distillation\\ column\end{tabular} & \begin{tabular}[c]{@{}l@{}}Diameter of plate column estimated using methodology proposed in Seider et al. 2009;\\ height to diameter ratio designed to be 25 in accordance with Douglas 1988\end{tabular} & \begin{tabular}[c]{@{}l@{}}Seider et al. 2009;\\ Douglas 1990\end{tabular} \\
S303            & \begin{tabular}[c]{@{}l@{}}Distillation\\ column\end{tabular} & \begin{tabular}[c]{@{}l@{}}Diameter of plate column estimated using methodology proposed in Seider et al. 2009;\\ height to diameter ratio designed to be 25 in accordance with Douglas 1988\end{tabular} & \begin{tabular}[c]{@{}l@{}}Seider et al. 2009;\\ Douglas 1991\end{tabular} \\
S401            & Flash drum                                                    & Minimum residence time of liquid phase is 5 min; vessel half-filled with liquid                                                                                                                           & Seader et al. 2011                                                         \\
S402            & Crystalliser                                                  & Scaling from past example (Sulzer falling-film melt crystalliser)                                                                                                                                         & Seader et al. 2012                                                         \\
S501            & Flash drum                                                    & Minimum residence time of liquid phase is 5 min; vessel half-filled with liquid                                                                                                                           & Seader et al. 2013                                                         \\
S502            & \begin{tabular}[c]{@{}l@{}}Distillation\\ column\end{tabular} & \begin{tabular}[c]{@{}l@{}}Diameter of plate column estimated using methodology proposed in Seider et al. 2009;\\ height to diameter ratio designed to be 25 in accordance with Douglas 1988\end{tabular} & \begin{tabular}[c]{@{}l@{}}Seider et al. 2009;\\ Douglas 1991\end{tabular} \\
S503            & \begin{tabular}[c]{@{}l@{}}Distillation\\ column\end{tabular} & \begin{tabular}[c]{@{}l@{}}Diameter of plate column estimated using methodology proposed in Seider et al. 2009;\\ height to diameter ratio designed to be 25 in accordance with Douglas 1988\end{tabular} & \begin{tabular}[c]{@{}l@{}}Seider et al. 2009;\\ Douglas 1992\end{tabular} \\ \hline
\end{tabular}%
}
\end{table}






\end{landscape}
