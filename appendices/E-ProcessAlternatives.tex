% !TeX root = ../main.tex
\section{Process alternatives}
\label{app:alternatives}

\subsection{Mixed-acid nitration}
\label{mixed}

\subsection{Nitration reactor}
\label{nitrationreactor}
\subsubsection{Packed-bed reactor}
A conventional packed-bed reactor was considered for the nitration process. It is easy to operate because the packed structure of the catalyst removes the need for catalyst filtration in the outlet stream . The compact catalyst packing also allows high catalyst loading []. A simple modelling of a packed-bed reactor usually assumes perfect mixing, however, in reality flow maldistribution and hot spot formation occurs []. This is dangerous for the nitration process as any hot spots promote the thermal runaway process that increases the risk of explosion. Therefore, to increase the inherent safety of Nitroma's plant, this option was not pursued further.

\subsubsection{Slurry reactor}

\subsection{Toluidine hydrogenation reactor}
\label{toluidine}
\subsubsection{Stirred Slurry reactor}
An alternative reactor for toluidine hydrogenation was the slurry reactor. Attrition of the solid Pd catalyst will occur in slurry reactors due to the violent agitation in the reactor, damaging the catalyst. A slurry reactor was not applicable for a high pressure operations like this due to the 


\subsection{Direct 4-aminobenzaldehyde production}
\label{direct}