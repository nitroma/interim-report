% !TeX root = ../main.tex
\section{Process alternatives}
\label{app:alternatives}

\subsection{Mixed-acid nitration}
Commercial production of nitrotoluene is commonly done using a mixture of nitric acid and sulfuric acid. To improve the interfacial area between the two phases formed, vigorous agitation is required []. 


\subsection{Direct production of 4-aminobenzaldehyde}


\subsection{Nitration reactor}

\subsubsection{Packed-bed reactor}
A conventional packed-bed reactor was considered for the nitration process. This reactor is easy to operate because the packed structure of the catalyst removes the need for catalyst filtration in the outlet stream . The compact catalyst packing also allows for high zeolite catalyst loading, thus speeding up the reaction \cite{kashid_microstructured_2009}. A simple modelling of a packed-bed reactor usually assumes perfect mixing, however, in reality flow maldistribution and hot spot formation occurs \cite{nguyen_flow_1994}. This is dangerous for the nitration process as any hot spots will promote the thermal runaway process that increases the risk of explosion. Conversely, the preferred packed-bed microreactors offer the benefits of a packed-bed reactor, while preventing the possibility of hot spot formation through its enhanced heat transfer ability. Therefore, to increase the inherent safety of Nitroma's plant, this option was not pursued further.

\subsubsection{Stirred slurry reactor}
Stirred slurry reactor has an automatic agitator that controls the inflow and outflow of the product. This provides a high degree of automation \cite{liu_nitration_2019}, which is particularly useful when the reaction temperature has to be carefully controlled. However, in the presence of solid H-mordenite catalyst, the mechanical agitation will erode the catalyst in the liquid suspension \cite{argyle_heterogeneous_2015}. The catalytic activity decreases over time and needs to be regenerated to achieve the same conversion as fresh catalyst. Further filtration process also needs to be carried out on the reactor effluent to retain the catalyst in the reactor. The drawbacks outweigh the benefits, so this reactor was not chosen.

\subsection{Toluidine hydrogenation reactor}

\subsubsection{Stirred Slurry reactor}
An alternative reactor for toluidine hydrogenation was the slurry reactor. The slurry reactor provides a steady agitation of the reaction, increasing the selectivity [mass transfer hydrogenation paper]. Attrition of the solid Pd catalyst will occur in slurry reactors due to the violent agitation in the reactor, damaging the catalyst. As Pd catalyst is considered as a precious metal, catalyst regeneration will be very expensive thus lowering the EP of this plant. Moreover, based on the kinetics of the oxidation reaction from Equation [label equation here], the rate constant is strongly dependent on the Pd catalyst weight, further proving the importance of low catalyst attrition [].
The slurry reactor is more suitable for gas-phased hydrogenation with the usage of $H_2$ gas at lower pressures, which is not applicable the current liquid-phase hydrogenation operating at high pressures (13atm). Therefore, the stirred slurry reactor option is eliminated. [trickle bed textbook]

\subsubsection{Counter-current trickle bed reactor}
After deciding on trickle bed reactor as the preferred reactor for toluidine hydrogenation, 

A counter-current trickle bed reactor was considered for the increased overall performance of the reactor and improved wetting of catalyst. [novel countercurrent]. Moreover, a counter-current trickle bed reactor provides the option for in situ separation for catalyst via catalytic distillation, especially for for by-products which may act as catalyst inhibitor. However, in this reaction,the product of o-toluidine shows no sign of catalyst inhibitor, thus disregarding the proposed benefit. The major disadvantage of a counter-current trickle bed reactor was the risk of flooding, where counter-current phase is reversed, caused by the strong interfacial friction between upward moving gas and downward moving liquid [prediction of flooding]. It was later decided that the risk of flooding has offset all benefits provided by counter-current flow, thus the counter-current flow trickle bed reactor was not chosen. 

\subsubsection{Fluidised Bed Reactor}

\subsection{P-Nitrotoluene oxidation reactor}


