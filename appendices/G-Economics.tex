% !TeX root = ../main.tex

\section{Economic analysis}
\label{app:economics}
\subsection{Capital cost estimation}

The method outlined by Bridgewater in XXXX is used to estimate the capital cost of the chemical plant.
\begin{equation}
    C=280000 \times N \times (\frac{Q}{s})^0.3
\end{equation}

where $C$ is capital cost in \$, $Q$ is  plant capacity in t yr, $s$ is reactor conversion and $N$ is number of functional units (equipment and ancillaries needed for a significant process step such as reaction or separation).\\
The calculation was first carried out for the BA scenario.
\begin{equation}
    C=280000 \times 15 \times (\frac{837.32}{0.27})^0.3
\end{equation}
\begin{equation}
    C=\$46.9m
\end{equation}
    
The calculation was then carried out for the BH scenario.
\begin{equation}
    C=280000 \times 17 \times (\frac{1208.69}{0.94} )^0.3
\end{equation}
\begin{equation}
    C=\$40.8m
\end{equation}

The larger value of \$46.9m was taken as the final estimate for Nitroma's chemical plant as it reflects the highest estimated capital expenditure from Nitroma's two operational procedures.