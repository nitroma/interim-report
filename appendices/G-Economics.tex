% !TeX root = ../main.tex
\section{Economic analysis}
\label{app:economics}
\subsection{Price of chemicals}

Early findings from a study of market prices of the chemicals used in Nitroma's plant can be found in \cref{tab:material-prices}. These values were used in the calculation of economic potential.

\begin{table}[h] 
\centering
\caption{Price of chemicals in China}
\label{tab:material-prices}
\begin{tabular}{lll}
    \toprule
    Material    & Price (\si{\USD\per\kg}) & Reference \\ \midrule
    Toluene     & 0.62          & \cite{sunsirs_commodity_group_china_2021}     \\
    Nitric acid & 0.31          & \cite{sun_sirs_commodity_group_china_2021}     \\
    Hydrogen    & 0.03          & \cite{sefic_gas_low_nodate}     \\
    Propanol    & 1.15          & \cite{jiangsu_yongtaihua_chemical_co_high_2019}     \\
    Formic acid & 0.36          & \cite{china_petroleum__and_chemical_industry_federation_china_2020}     \\ 
    4-ABA       & 26.29         & \cite{xian_baiqing_biotech_co_professional_2020}     \\
    4-ABH       & 19.00         & XX     \\
    o-TOL       & 10.26         & XX    \\\bottomrule
\end{tabular}
\end{table}

\subsection{Economic potential: Level 1}
The economic potential of the plant was found using the following basic equations:
\begin{align}
    \text{Economic potential} &= \text{Revenue from products} - \text{Cost of raw materials} \label{eq:ep1}\\
    \text{Revenue from products} &= \sum_{i=1}^{3} \text{Mass flow rate of product X} \cdot \text{Price of product X} \label{eq:ep1-revenue}\\
    \text{Cost of raw materials} &= \sum_{i=1}^{5} \text{Mass flow rate of feed X} \cdot \text{Price of feed X} \label{eq:ep1-cost}
\end{align}
\\
From the Aspen simulation of Nitroma's chemical plant, the following data regarding the feed and product streams was acquired:
\begin{table}[h] 
\centering
\caption{Feed and product flow rates}
\label{tab:costing-flows}
\begin{tabular}{llS[table-format=7.2]S[table-format=7.2]}
\toprule
Stream type & Material    & \multicolumn{2}{l}{Mass flow rate (\si{\kg\per\year})} \\ \cmidrule(l){3-4}
            &             & {ABH scenario}            & {ABA scenario}     \\ \midrule % these cells had to be wrapped in {} so that the number parser doesn't get confused -- it interprets the 'e' character as an exponent as in 5.5e-6  Ahh i see
Feed 1      & Toluene     & 1256220.56        & 1256215.43     \\
Feed 2      & Nitric acid & 1186103.30         & 1186028.03     \\
Feed 3      & Hydrogen    & 507628.78         & 507464.15       \\
Feed 4      & Propanol    & 3488746.39        & 3490804.44      \\
Feed 5      & Formic acid & 29286.60          & 137648.17        \\
Product 1   & 4-ABA       & 78070.00          & 112770.00        \\
Product 2   & 4-ABH       & 406032.01         & 0.00              \\
Product 3   & o-TOL       & 724551.26         & 724551.26        \\ \bottomrule
\end{tabular}
\end{table}

Substituting values from \cref{tab:material-prices,tab:costing-flows} into \cref{eq:ep1,eq:ep1-revenue,eq:ep1-cost} yields an economic potential of \$12.0m per year for the ABH scenario and \$5.2m per year for the ABA scenario. The percentage of the plant's annual running time under each scenario can vary from 0-100\%, as long as the sum of running times of both scenarios add up to \SI{8000}{\hour}. This means that Nitroma's annual profitability can vary from \$5.2m-\$12.0m. It should be noted that key operating costs (e.g. utilities, labour, maintenance, taxes) and key by-product revenues (e.g. \textit{para}-nitrotoluene) have been neglected at this stage.

\subsection{Capital cost estimation}
Bridgewater's method outlined in Principles, Practices and Economics of Plant Design is used to estimate the capital cost of the chemical plant (XX). This uses the following equation:
\begin{equation}
    C= \num{280000} \times N \times \left(\frac{Q}{s}\right)^{0.3}
\end{equation}
where $C$ is capital cost in \$, $Q$ is  plant capacity in \si{\tonne\per\year}, $s$ is reactor conversion and $N$ is number of functional units (equipment and ancillaries needed for a significant process step such as reaction or separation).

The calculation was first carried out for the ABH scenario.
\begin{align*}
C &= 280000 \times 17 \times \left(\frac{1208.69}{0.94}\right)^{0.3}  \\
  &=\$40.8m 
\end{align*}

The calculation was then carried out for the ABA scenario.
\begin{align*}
C &= \num{280000} \times 15 \times \left(\frac{837.32}{0.27}\right)^{0.3}  \\
  &= \$46.9m 
\end{align*}

The larger value of \$46.9m was taken as the final estimate for Nitroma's chemical plant as it reflects the highest predicted capital cost from Nitroma's two operational scenarios.