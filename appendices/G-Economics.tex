% !TeX root = ../main.tex
\section{Economic analysis}
\label{app:economics}
\subsection{Price of chemicals}

Early findings from a study of market prices of the chemicals used in Nitroma's plant are found in \cref{tab:material-prices}. These values were used in the calculation of economic potential.

\begin{table}[h] 
\centering
\caption{Price of chemicals in China}
\label{tab:material-prices}
\begin{tabular}{lll}
    \toprule
    Material    & Price (\$/kg) & Reference \\ \midrule
    Toluene     & 0.62          & XX        \\
    Nitric acid & 0.31          & XX        \\
    Hydrogen    & 0.03         & XX        \\
    Propanol    & 1.15          & XX        \\
    Formic acid & 0.36          & XX        \\ \bottomrule
\end{tabular}
\end{table}

\subsection{Economic potential: Level 1}
The economic potential of the plant was found using the following basic equation:
\begin{equation}
    \text{Economic potential} = \text{Revenue from products} - \text{Cost of raw materials}
\end{equation}
\begin{equation}
    \text{Economic potential} = (\text{Mass flow rate of product X} \times \text{Price of product X}) - (\text{Mass flow rate of raw material Y} \times \text{Price of Y})
\end{equation}
\\
From the Aspen simulation of Nitroma'a chemical plant, the following data regarding the feed and product streams was acquired:

\begin{table}[h] 
\centering
\caption{klow rates}
\label{tab:material-prices}
\begin{tabular}{lll}
    \toprule
    Stream type    & Material    & \multicolumn{2}{c|}{Mass flow rate (\$/kg)}    \\\midrule
    & \splitcell{} & \splitcell{} & {\splitcell{ABH scenario} & \splitcell{ABA scenario}}\midrule
    Feed     & Toluene     & 0.62                     \\
    Feed    & Nitric acid & 0.31                      \\
    Feed    & Hydrogen    & 0.034                     \\
    Feed    & Propanol    & 1.15                      \\
    Feed    & Formic acid & 0.36                      \\ 
    Product & 4-ABH       & 0.36                      \\
    Product & 4-ABA       & 0.36                      \\
    Product & o-TOL       & 0.36                      \\\bottomrule
\end{tabular}
\end{table}

\subsection{Capital cost estimation}

Bridgewater's method outlined in Principles, Practices and Economics of Plant Design is used to estimate the capital cost of the chemical plant (XX).
\begin{equation}
    C= \num{280000} \times N \times \left(\frac{Q}{s}\right)^{0.3}
\end{equation}
where $C$ is capital cost in \$, $Q$ is  plant capacity in t yr, $s$ is reactor conversion and $N$ is number of functional units (equipment and ancillaries needed for a significant process step such as reaction or separation).

The calculation was first carried out for the BA scenario.
\begin{align*}
C &= \num{280000} \times 15 \times \left(\frac{837.32}{0.27}\right)^{0.3}  \\
  &= \$46.9m 
\end{align*}
 \\
The calculation was then carried out for the BH scenario.
\begin{align*}
C &= 280000 \times 17 \times \left(\frac{1208.69}{0.94}\right)^{0.3}  \\
  &=\$40.8m 
\end{align*}

The larger value of \$46.9m was taken as the final capital estimate for Nitroma's chemical plant as it reflects the highest estimated capital cost from Nitroma's two operational procedures.