% !TeX root = ../main.tex
\section{Synthesis}
\label{sec:synthesis}
\subsection{Decision analysis}
%explain TOPSIS and AHP
To make systematic and informed decisions for key aspects of the project, multi-criteria decision making (MCDM) methods were employed. Firstly, pairwise comparison of the criteria was performed with the Analytical Hierarchy Process (AHP) to produce criteria weightings which were fed into the Technique for Order of Preference by Similarity to Ideal Solution (TOPSIS) analysis. Processing of all relevant information though AHP and TOPSIS analysis yielded quantitative rankings for process alternatives, which were initially short-listed with qualitative arguments. Greater importance was given to safety and environmental concerns, technical performance and economic potential.
This methodology was applied to the following key choices: three products selection, plant location, toluene nitration synthesis, hydrogenation solvent and XXX 

\subsection{Product selection}
To deliver a multi-purpose plant, Nitroma selected toluene as feedstock due to the industrial importance of its nitration products and their amino derivatives. All three isomeric nitrotoluenes can indeed easily be oxidised and reduced to substituted aromatic amides of significant importance for the pharmaceutical, agrochemical and textile industries []. Six chemicals were initially selected due to their applications in drugs, fertilizers or dyes: 2-aminobenzaldehyde (), 4-aminobenzaldehyde (), 4-aminobenzoic acid (), o-toluidine (), m-toluidine () and p-toluidine (). To determine the three most advantageous products to produce, AHP/TOPSIS analysis was conducted, giving significant importance to the safety of the compounds and the economic potential. Safety was assessed using the NFPA Hazard Classification for health, flammability and instability, where the smallest value is preferable. The economic potential of each chemical was determined using teh average price, the average market share of producers (representing the share Nitroma can plan to cap\Cref{tab:product} summarises the results and the complete methodology can be found in \Cref{app:matrix}.   

% Please add the following required packages to your document preamble:
% \usepackage{multirow}
% \usepackage[table,xcdraw]{xcolor}

\begin{table}[H]
\centering
    \caption{AHP/TOPSIS results for product selection}
    \label{tab:product}\footnotesize
\begin{tabularx}{\linewidth}{l|XXXX|XX|lll|XXX}
\toprule
                                          & \multicolumn{4}{c}{Economic potential   (29\%)}                                & \multicolumn{2}{|p{2cm}}{Process   complexity (14\%)} & \multicolumn{3}{|c|}{EHS (57\%)}     &                       &                          &                           \\ \cmidrule(r){2-5} \cmidrule(r){6-7} \cmidrule(r){8-10}
                                          & Price (\$/kg) & CAGR  & Average market share of producer & Demand (tonnes/year) & Number of reactions & Max selectivity to toluene & \rotatebox[origin=l]{270}{Health} & \rotatebox[origin=l]{270}{Flammability} & \rotatebox[origin=l]{270}{Instability} & AHP & TOPSIS & Rank \\ \midrule
2-aminobenzaldehyde & 1711          & 5.6\% & 0.77\%                           & 1344                 & 0.5                 & 0.53                       & 2      & 1            & 1           & 0.124                 & 0.273                    & 6                         \\ 
4-aminobenzaldehyde & 19            & 7.1\% & 1.75\%                           & 336842               & 0.5                 & 0.44                       & 2      & 1            & 0           & 0.188                 & 0.563                    & \cellcolor{green}2 \\ 
o-toluidine         & 5.8           & 7.8\% & 0.88\%                           & 1327586              & 1                   & 0.53                       & 3      & 2            & 0           & 0.210                 & 0.611                    & \cellcolor{green}1 \\ 
p-toluidine         & 9             & 6.3\% & 0.50\%                           & 722222               & 1                   & 0.44                       & 3      & 2            & 0           & 0.150                 & 0.333                    & 5                         \\ 
m-toluidine         & 8.9           & 8.2\% & 0.57\%                           & 1067416              & 1                   & 0.03                       & 4      & 2            & 0           & 0.161                 & 0.458                    & 4                         \\ 
4-aminobenzoic acid & 26.29         & 5.8\% & 3.70\%                           & 311906               & 0.5                 & 0.44                       & 2      & 1            & 1           & 0.168                 & 0.478                    & \cellcolor{green}3 \\ \bottomrule
\end{tabularx}
\end{table}

Due to their high economic potential and reduced health and fire hazards, 4-aminobenzaldehyde, o-toluidine and 4-aminobenzoic acid were found to be the most valuable compounds to be produced in Nitroma's multi-purpose plant.  A visual summary of the selected compounds and their associated synthesis pathways can be found in \Cref{app:matrix}.



\subsection{Nitration method}
\subsubsection{Choice of nitrating agent}
Nitric acid has been selected as the source of nitrogen for the nitration of toluene due to its safe and environmentally friendly nature compared to other possible nitrating agents []. Although nitric acid is a highly acidic and volatile compound, compared to alternatives such as acetone cyanohydrin and dinitrogen tetroxide which are highly toxic, nitric acid is more appropriate for industrial-scale nitration of toluene []. Nitric acid, introduced as an aqueous solution, is indeed the most commonly used nitrating agent for this process in industry. (cite Ullmann's toluidine) A common alternative to nitric acid is acetyl nitrate, which is formed by the reaction of nitric acid with acetic anhydride. This reaction also yields methanoic acid as by-product, thus resulting in a lower atom economy []. It will also cause extra difficulties in separations downstream by introducing more components into the process. The same argument can be employed for other alkyl nitrates such as butyl nitrate. 



\begin{comment}
Nitric acid has been selected as the source of nitrogen for the nitration of toluene because of its various advantages.

Industrially, nitric acid is the most commonly used nitrating agent for this process. (cite Ullmann's toluidine) It is thus also the most well-studied in the literature. Usually, it is introduced into the process as an aqueous solution. As some studies have suggested, one of the distinct advantages of nitric acid is that due to its strong acidity, it can act as a self-catalyst by self-donation of proton during the nitration of toluene, (cite that nice paper), though a strong acidic catalyst such as sulfuric acid or zeolite is nonetheless recommended in a typical process.

A common alternative to nitric acid for toluene nitration is acetyl nitrate. Its introduction into the process is usually done as a mixture of nitric acid and acetic anhydride, which forms acetyl nitrate upon reaction and methanoic acid as by-product. Compared to employing just nitric acid as the nitrating agent, we consider the use of acetyl nitrate as unnecessary as it is not atom economic and it will cause extra difficulties in separations downstream by introducing more components into the process. The same argument can be employed for other alkyl nitrates such as butyl nitrate. 

Although nitric acid is not generally considered a safe and environmentally friendly chemical due to its high acidity and volatility, an important advantage of it compared to many other nitrating agents is that it is much safer and less environmentally damaging. Alternatives like acetone cyanohydrin and dinitrogen tetroxide are highly toxic and not ideal for industrial-scale nitration of toluene. Nitric acid is relatively easy to treat and handle from both environmental and safety perspectives. 

Nitric acid has been selected as the source of nitrogen for the nitration of toluene among the various candidates. Although nitric acid is a highly acidic and volatile compound, [] compared to alternatives such as acetone cyanohydrin and dinitrogen tetroxide which are highly toxic, nitric acid is much safer and more environmentally friendly. [] 
Industrially, nitric acid is the most commonly used nitrating agent for the nitration of toluene. (cite Ullmann's toluidine) It is typically introduced into the process as an aqueous solution. As some studies have suggested, one of the distinct advantages of nitric acid is that due to its strong acidity, it can act as a self-catalyst by self-donation of proton during the nitration of toluene, (cite that nice paper), though a strong acidic catalyst such as sulfuric acid or zeolite is nevertheless recommended in a typical process. A common alternative to nitric acid is acetyl nitrate, which is formed by the reaction of nitric acid with acetic anhydride. But compared to having just nitric acid as the nitrating agent, this reaction causes extra difficulties in separations downstream by introducing more components into the process; it also yields methanoic acid as by-product, which results in a lower atom economy []. The same argument can be employed for other alkyl nitrates such as butyl nitrate. 
\end{comment}

\subsubsection{Choice of catalyst}
%assumptions
%catalyst selection process --> criteria + AHP/TOPSIS results
%kinetics derivation

\subsection{Oxidation and hydrogenation pathways}
\subsubsection{o-toluidine production}

\subsubsection{4-nitrotoluene oxidation}

\subsubsection{4-aminobenzoic acid synthesis}

\subsubsection{Preparation of 4-aminobenzaldehyde}
%assumptions
%catalyst/solvent/reductant/oxidant selection process --> criteria + AHP/TOPSIS results
%kinetics derivation

