% !TeX root = ../main.tex
\section{Process description}
\label{sec:process}
\subsection{Nitration step}
%reactor design
%selection criteria
%innovation
In line with Nitroma's goal of transitioning from batch to continuous process, packed-bed microreactors were chosen for the nitration of toluene. Due to their high surface area to volume ratio, microreactors are superior in heat transfer as compared to conventional reactors []. This was an important criteria when choosing nitration reactors as nitration is a highly exothermic process with possibility of thermal runaway if temperature is not controlled. Therefore, microreactor reduces the risk of thermal runaway that are common in highly exothermic nitration reactions. Additionally, 

While most There has been an increasing interest in deploying microreactors on large scale nitration processes, therefore Nitroma is poised to gain the first mover advantage by 
(Alternative continuous reactors were considered in Appendix [ ])
The catalysts in the microreactors were 
\subsection{Nitrotoluene isomers separation}



\subsection{o-toluidine production}
%trickle bed reactor
%Pd/c catalyst
For the o-toluidine hydrogenation, a cocurrent trickle bed reactor was selected
with both gas and liquid reactants flowing downwards. Trickle bed reactor was chosen due to the ease of operation at high pressure (13 atm) and the relative slow catalyst deactivation, which is imperative for an expensive catalyst usage of Pd. Moreover, the plug flow 


%assumptions to be added in appendix
\subsection{4-nitrotoluene oxidation}

\subsection{4-nitrobenzoic acid and 4-nitrobenzaldehyde hydrogenation}


\subsection{Major equipment sizing}


\subsection{Process alternatives}



%nitration synthesis
%2nd best reactor type
%direct or indirect 4-amino production