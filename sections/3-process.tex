% !TeX root = ../main.tex
\section{Process description}
\label{sec:process}
\subsection{Nitration step}
%reactor design
%selection criteria
%innovation

\subsection{Nitrotoluene isomers separation}



\subsection{o-toluidine production}

\subsection{4-nitrotoluene oxidation}

\subsection{4-nitrobenzoic acid and 4-nitrobenzaldehyde hydrogenation}


\subsection{Major equipment sizing}


\subsection{Process alternatives}
\subsection{}
A common alternative to nitric acid is acetyl nitrate which is formed by the reaction of nitric acid with acetic anhydride \cite{vassena_selective_1999}. This reaction yields formic acid as by-product, resulting in a lower atom economy; it also causes unnecessary difficulties in separations downstream by introducing three extra components: acetyl nitrate, acetic anhydride, and formic acid. The same argument can be employed for other alkyl nitrates such as butyl nitrate. To this end, nitric acid is deemed as the most suitable choice among all candidates.

%nitration synthesis
%2nd best reactor type
%direct or indirect 4-amino production