% !TeX root = ../main.tex
\section{Process description}
\label{sec:process}
\subsection{Nitration step}
%reactor design
%selection criteria
%innovation
In line with Nitroma's goal of transitioning from batch to continuous process, packed-bed microreactors were chosen for the nitration of toluene. Due to their high surface area to volume ratio, microreactors are superior in heat transfer as compared to conventional batch or semi-batch reactors \cite{halder_nitration_2007}. This was an important criteria when choosing nitration reactors as nitration is a highly exothermic process with the hazard of thermal runaway if the temperature is not controlled. Additionally, the high mass transfer rate within the microreactors prevent the formation of by-products \cite{halder_nitration_2007} which complicates the separation process downstream.
(Alternative continuous reactors that were considered are detailed in Appendix \ref{nitrationreactor}.)

There has been an increasing interest in deploying microreactors on large scale nitration processes, but most industrial nitration reactors are still limited to batch and semi-batch processes. Therefore, Nitroma is poised to gain the first mover advantage by scaling up safe, highly-efficient, and innovative microreactors capable of handling hundreds of tons of throughput per year.

Toluene (95\% molar basis) and nitric acid (50\% molar basis) are fed to the isothermal packed-bed microreactor operating at 333K, 1 atm, and 90\% conversion. The reaction produces 3 isomers of nitrotoluene --- o-nitrotoluene (ONT), m-nitrotoluene (MNT) and p-nitrotoluene (PNT) --- at a selectivity (\%) of 53:4:44 respectively. The reactor effluent consists of the 3 isomers along with nitric acid, water and unreacted toluene that are fed to the next stage of the process.

\subsection{Nitrotoluene isomers separation}
Despite an optimised reactor, the 3 isomers of nitrotoluene produced need to be separated. The reactor effluent first passes through a decanter (S101) to separate the aqueous phase of nitric acid and water from the organic phase. The nitric acid is concentrated in a distillation column (S102) by boiling off water and recycled back into the reactor to be reused. Meanwhile, the organic phase is sent to a distillation column (S103) that separates toluene from the nitrotoluenes with heavier boiling points. The unreacted toluene can then be recycled back into the nitration reactor. The nitrotoluenes are sent to another distillation column (S201) to separate most of o-nitrotoluene from m and p-nitrotoluene. The p-nitrotoluene is then separated from m-nitrotoluene via a continuous crystalliser (S202) and centrifuge (S203). As this crucial separation determines the production rate of our downstream products, careful selection of the separation techniques was done using a quantitative approach called the Jaksland analysis [] (Details in Appendix [ ]). Distillation-crystallisation ranked as the best separation technique due to the large difference in boiling points between o- and p-nitrotoluene, and the large difference in melting points between m- and p-nitrotoluene.

\subsection{Production of o-toluidine}
%trickle bed reactor
%Pd/c catalyst
%add flowrate, temperature, pressure, what phases,  isothermal/adiabatically
The o-nitrotoluene is then hydrogenated to o-toluidine in a co-current trickle bed reactor with both gas and liquid reactants flowing downwards. Trickle bed reactor was chosen due to the ease of operation at high pressure (13 atm) and the relative slow catalyst deactivation, which is imperative for an expensive catalyst usage of Pd. Co-current flow is selected due to the reduced risk of flooding, thus allowing for large flow of products[]. Moreover, the liquid plug flow behaviour in trickle bed reactor allows for a higher conversion than other reactors, matching the 90\% conversion rate target of o-toluidine by Nitroma. 

o-toluidine is then purified to meet the purity requirement of 98\% through a sequence of separators. A flash drum (S501) is first used to remove excess hydrogen gas. A distillation column (S502) then removes the propanol and water from the heavier organic compounds. Finally, a vacuum distillation column (S503) removes most of the unreacted o-nitrotoluene, yielding a >99\% pure o-toluidine stream. The decreased pressure lowers the o-nitrotoluene boiling point more than o-toluidine, making distillation easier and more efficient.  
 
%assumptions to be added in appendix
\subsection{Production of 4-nitrobenzaldehyde and 4-nitrobenzoic acid}
The crystallised 4-nitrotoluene is dissolved in a methanol solvent, and then fed into a reactor to be oxidised with air. The operating conditions of the reactor are optimised to either a partial oxidation to produce 4-aminobenzaldehyde or complete oxidation to product 4-aminobenzoic acid, depending on which product is desired. Both the oxidation reactors are operated in  

In both cases, the reactor effluent is sent to a flash drum (S301) that removes air, vapourised methanol and water from the organics stream. In this preliminary design, the methanol is treated as a waste stream, although it is likely that this will be recycled in the final design. The organic stream is then fed into a distillation column (S302) and purified before the hydrogenation reaction further downstream.

If the reactor is optimised to a complete oxidation to produce 4-nitrobenzoic acid, the organic stream contains 4-nitrotoluene, 4-nitrobenzaldehyde and 4-nitrobenzoic acid. The two lighter components are distilled off, leaving a high purity 4-nitrobenzoic acid as the bottoms stream. If the reactor is optimsed to a partial oxidation to produce 4-nitrobenzaldehyde, the organic stream that feeds into the second distillation column will only contain 4-nitrotoluene and 4-nitrobenzaldehyde, which will be the top and bottoms product respectively. 

\subsection{Hydrogenation of 4-aminobenzaldehyde and 4-aminobenzoic acid}

A packed-bed reactor was selected for the final hydrogenation process with 
The effluent from the reactor is then purified to yield product streams that meet the 98\% purity requirement in the negotiated brief. The separation unit is designed to be able to purify both 4-aminobenzaldehyde and 4-aminobenzoic acid to >98\% purity.

\subsection{Major equipment sizing}


\subsection{Process alternatives}
During the design, several options were considered for different sections of the process. The main alternative for nitration is the process with mixed-acid instead of Nitroma's heterogeneous reaction process. The direct production of 4-aminobenzaldehyde from 4-nitrotoluene was also examined. Furthermore, multiple reactor types were envisioned for nitration, oxidation and hydrogenation. Details of those process alternatives can be found in \Cref{app:alternatives}.

%to be added in appendix? --> just mention them here and refer to appendix but we need it here too


%nitration synthesis --> with mixed acid
%funky intensified reactors
%2nd best reactor type --> 
%direct or indirect 4-amino production --> one pot 