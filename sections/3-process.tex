% !TeX root = ../main.tex
\section{Process description}
\label{sec:process}
\subsection{Nitration step}
%reactor design
%selection criteria
%innovation
In line with Nitroma's goal of transitioning from batch to continuous process, packed-bed microreactors were chosen for the nitration of toluene. Due to their high surface area to volume ratio, microreactors are superior in heat transfer as compared to conventional batch or semi-batch reactors []. This was an important criteria when choosing nitration reactors as nitration is a highly exothermic process with the hazard of thermal runaway if the temperature is not controlled. Additionally, the high mass transfer rate within the microreactors prevent the formation of by-products which complicates the separation process downstream.
(Alternative continuous reactors were considered in Appendix [ ])

There has been an increasing interest in deploying microreactors on large scale nitration processes, but most industrial nitration reactors are still limited to batch and semi-batch processes. Therefore, Nitroma is poised to gain the first mover advantage by designing safe, highly-efficient, and innovative microreactors capable of handling hundreds of tons of throughput per year.

\subsection{Nitrotoluene isomers separation}
Despite an optimised reactor, 3 isomers of nitrotoluene are produced which need to be separated. The reactor effluent first passes through a decanter (S101) to separate the aqueous phase of nitric acid and water from the organic phase. The nitric acid is concentrated in a distillation column (S102) by boiling off water and recycled back into the reactor to be reused. Meanwhile, the organic phase is sent to a distillation column (S103) to separate o-nitrotoluene from m and p-nitrotoluene. The p-nitrotoluene is then separated from m-nitrotoluene via a continuous crystalliser and subsequent centrifugation. As this crucial separation determines the production rate of our downstream products, careful selection of the separation techniques was done using a quantitative approach called the Jaksland analysis [] (Details in Appendix [ ]). Distillation-crystallisation ranked as the best separation technique due to the large difference in boiling points between o- and p-nitrotoluene, and the large difference in melting points between m- and p-nitrotoluene.

\subsection{o-toluidine production}
%trickle bed reactor
%Pd/c catalyst
For the o-toluidine hydrogenation, a cocurrent trickle bed reactor was selected
with both gas and liquid reactants flowing downwards. Trickle bed reactor was chosen due to the ease of operation at high pressure (13 atm) and the relative slow catalyst deactivation, which is imperative for an expensive catalyst usage of Pd. Moreover, the plug flow liquid flow in trickle bed reactor allows for a higher conversion than other reactors, matching  the 90\% conversion rate target of o-toluidine by Nitroma. 


%assumptions to be added in appendix
\subsection{4-nitrotoluene oxidation}

\subsection{4-nitrobenzoic acid and 4-nitrobenzaldehyde hydrogenation}


\subsection{Major equipment sizing}
%to be added in appendix?

\subsection{Process alternatives}
%to be added in appendix? --> just mention them here and refer to appendix but we need it here too


%nitration synthesis --> with mixed acid
%funky intensified reactors
%2nd best reactor type --> 
%direct or indirect 4-amino production --> one pot 