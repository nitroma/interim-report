% !TeX root = ../main.tex
\section{Process description}
\label{sec:process}
\subsection{Nitration step}
%reactor design
%selection criteria
%innovation
In line with Nitroma's goal of transitioning from batch to continuous process, packed-bed microreactors are chosen for the nitration of toluene. Due to their high surface area to volume ratio, microreactors are superior in heat transfer as compared to conventional batch or semi-batch reactors \cite{halder_nitration_2007}. This was an important criteria when choosing nitration reactors as nitration is a highly exothermic process with the hazard of thermal runaway if the temperature is not controlled. Additionally, the high mass transfer rate within the microreactors prevent the formation of by-products \cite{halder_nitration_2007} which complicates the separation process downstream.
(Alternative continuous reactors that were considered are detailed in Appendix \ref{nitrationreactor}.)

There has been an increasing interest in deploying microreactors on large scale nitration processes, but most industrial nitration reactors are still limited to batch and semi-batch processes. Therefore, Nitroma is poised to gain the first mover advantage by scaling up safe, highly-efficient, and novel microreactors capable of handling hundreds of tons of throughput per year.

Toluene (95\% molar basis) and nitric acid (50\% molar basis) are fed to the isothermal packed-bed microreactor operating at 333K, 1 atm, and 90\% conversion. The reaction produces 3 isomers of nitrotoluene --- o-nitrotoluene (ONT), m-nitrotoluene (MNT) and p-nitrotoluene (PNT) --- at a selectivity ratio of 53:4:44 respectively. The reactor effluent consists of the 3 isomers along with nitric acid, water and unreacted toluene that are subsequently separated downstream.
\nomenclature[A]{MNT}{3-nitrotoluene, $meta$ isomer}

\subsection{Nitrotoluene isomers separation}
The separation of the reactor effluent begins as it is fed to a decanter (S101) to separate the aqueous phase of nitric acid and water from the organic phase. The nitric acid is recycled back into the nitration reactor at a 94\% (mole) purity after being concentrated in a distillation column (S102) that boils off water.

Meanwhile, the organic phase is sent to a distillation column (S103) that recovers 99.8 mol\% of the toluene from the less volatile nitrotoluenes, and fed back to the nitration reactor. The nitrotoluenes are sent to another distillation column (S201) to separate 99.99 mol\% of the more volatile ONT from MNT and PNT. The large difference in the melting points of the PNT and MNT is exploited in the separation of the isomers in a continuous falling-film crystalliser (S202). 

The nitrotoluene isomer separation is a crucial step, as it determines the production rate of downstream products. Thus the separation techniques above were carefully selected using both a quantitative method (Jakskand analysis)[] and sound engineering judgement (Details in Appendix [ ]). Operating conditions of each separation unit can also be found in Appendix []. 

\subsection{Production of o-toluidine}
%trickle bed reactor
%Pd/c catalyst
%add flowrate, temperature, pressure, what phases,  isothermal/adiabatically
The ONT is then hydrogenated to o-TOL in a co-current trickle bed reactor with both gas and liquid reactants flowing downwards. Trickle bed reactor was chosen due to the ease of operation at high pressure (13 atm) and the relative slow catalyst deactivation, which is imperative for an expensive catalyst usage of Pd \cite{vemala_hydrodynamic_nodate}. Co-current flow is selected due to the reduced risk of flooding, thus allowing for large flow of products[]. Moreover, the liquid plug flow behaviour in trickle bed reactor allows for a higher conversion than other reactors, matching the 90\% conversion rate target of o-TOL by Nitroma. 

o-TOL is then purified to meet the purity requirement of 98mol\%. A flash drum (S501) is first used to remove excess hydrogen gas from the reactor effluent. A distillation column (S502) then removes the propanol and water from the less volatile organic compounds. Finally, a second distillation column (S503) removes 94.8mol\% of unreacted ONT, yielding a 99.4mol\% pure o-TOL product.
 
%assumptions to be added in appendix
\subsection{Production of 4-nitrobenzaldehyde and 4-nitrobenzoic acid}
The crystallised PNT is dissolved in a methanol solvent, and then fed into a packed-bed reactor to be oxidised with air. [mention reaction conditions] The operating conditions of the reactor are optimised to either partial oxidation to produce 4-NBH or complete oxidation to product 4-NBA, depending on which product is desired.  

The oxidation reactors are operated in parallel, with each of the reactors producing their respective main products (4-NBH or 4-NBA). Future work will be done to redesign the oxidation reactors to be operated in series to reduce the loss of unreacted reactants.

The effluent from both reactors are sent to a flash drum (S301) that removes air and water vapour from the organics stream. The organic stream is then fed into a distillation column (S302), where 99.9mol\% of the unreacted PNT is separated from the less volatile nitroaromatics. 

Processing the effluent from the 4-NBA reactor in column S302 will also result in 99mol\% of 4-NBH as well as water being removed with the PNT. These chemicals are separated in a second distillation column (S303), yielding a 93mol\% pure PNT stream which can be sold as a by-product and a 58.9mol\% pure 4-NBH stream. Future work will recycle the 4-NBH to the oxidation reactor to be converted to 4-NBA.

The effluent from the 4-NBH reactor contains a 9.7mol\% of 4-NBA, of which 99.9 mol\% is separated from the more volatile compounds in S302. The lighter PNT,  4-NBA and water are separated in column S303, yielding a 99.9mol\% pure 4-NBA stream and a waste stream containing PNT and water. Future work will recycle the unreacted PNT into the oxidation reactor.

\subsection{Hydrogenation to 4-aminobenzaldehyde and 4-aminobenzoic acid}

Finally, both 4-ABH and 4-ABA is hydrogenated into two separate packed-bed reactors**, using formic acid as transfer hydrogen donor and Pd metal as catalyst. The packed-bed reactor is selected due to the control and better heat transfer ability and mixing. Moreover, the ability of continuous flow hydrogenation in packed-bed reactor allows for a good recovery of catalyst, where investigations has shown that recycled catalyst have similar efficiency as a fresh catalyst. This is of paramount importance for this reactor as the constant regeneration of expensive Pd catalyst greatly decrease the EP of this reaction [green hydro]. 

The amino products are separately purified. The effluent from the 4-ABH reactor is sent to a flash (S601) to remove air and water vapour.  Then, two distillation columns are employed to remove unreacted 4-nitrobenzaldehyde and remaining water to yield a 99.9mol\% purified 4-ABH. 97mol\% of the 4-ABH produced in the reactor can be recovered at the end of the separation.

The effluent from the 4-ABA reactor is sent to a flash (S601) to remove air. The remaining liquid is then sent to a continuous falling-film crystalliser to obtain pure 4-ABA crystals. It was decided that due to the large amount of water present, a distillation column would require a high heat duty to separate the compounds. The crystalliser is able to yield a 96mol\% recovery of 4-ABA from the reactor.

\subsection{Major equipment sizing}


\subsection{Process alternatives}
During the design, several options were considered for different sections of the process. The main alternative for nitration is the process with mixed-acid instead of Nitroma's heterogeneous reaction process. The direct production of 4-ABH from PNT was also examined. Furthermore, multiple reactor types were envisioned for nitration, oxidation and hydrogenation. Details of those process alternatives can be found in \Cref{app:alternatives}.

The initial design process also considered alternative separation techniques such as absorption, adsorption, microfiltration and membrane separation. While these techniques are relatively novel compared to those employed in the current design, they were eventually discarded due to the main reasons of being less reliable in their operation over long periods of time, and having higher capital cost []. Full details can be seen in Appendix [].

%to be added in appendix? --> just mention them here and refer to appendix but we need it here too


%nitration synthesis --> with mixed acid
%funky intensified reactors
%2nd best reactor type --> 
%direct or indirect 4-amino production --> one pot 