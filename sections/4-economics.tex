% !TeX root = ../main.tex
\section{Business assessment}
\label{sec:economics}
\subsection{Selection of location} 
AHP and TOPSIS methodologies were used to determine a suitable country for Nitroma’s chemical plant. The key factors taken into consideration, measured through various metrics, are summarised in Table XX.

China was identified as the optimal country for the chemical plant, owing to its strong local supply chain and low business operating costs. A sensitivity analysis was performed to understand the effects of changes to the metrics in Table XX, but none of the scenarios downgraded China. To identify an optimal city within China, the spread of toluene suppliers, access to distribution channels, local market demand, business policies and skilled labour availability across the provinces of China were studied. The Nanjing Jiangbei New Material Science Park located in Nanjing, Jiangsu was selected. China’s largest toluene manufactory, operated by Sinopec Yangzi Petrochemical, is situated within 10 km of Nitroma’s selected site in Jiangsu. The Jiangsu province also hosts 20\% of China’s inland waterways (22,500 km) and 4,710 km of highways, creating reliable access to the neighbouring Shanghai Port and large markets in Shandong and Zhegiang. Moreover, Jiangsu operates a favourable policy towards foreign-owned and private businesses (XX). 

\subsection{Production Capacity}


\subsection{Business model} 

\subsection{Market analysis}

\subsubsection{Feedstock Supply}
Toluene is used in construction, explosions and agrochemicals which drives the domestic demand for it in China []. The price of toluene is susceptible to fluctuating prices in crude oil which leads to fluctuating global supply and demand []. China had plant expansions in 2019 [] therefore domestic production of toluene is expected to increase and provide reliable supply[]. Toluene will be sourced from Sinopec Yangzi Petrochemical in Jiangsu, China[].

\subsection{Economic analysis} 