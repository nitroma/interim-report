% !TeX root = ../main.tex
\section{Business assessment}
\label{sec:economics}
\subsection{Selection of location} 
AHP and TOPSIS methodologies were used to determine a suitable country for Nitroma’s chemical plant. The key factors taken into consideration, measured through various metrics, are summarised in Table XX.

China was identified as the optimal country for the chemical plant, owing to its strong local supply chain and low business operating costs. A sensitivity analysis was performed to understand the effects of changes to the metrics in Table XX, but none of the scenarios downgraded China. To identify an optimal city within China, the spread of toluene suppliers, access to distribution channels, local market demand, business policies and skilled labour availability across the provinces of China were studied. The Nanjing Jiangbei New Material Science Park located in Nanjing, Jiangsu was selected. China’s largest toluene manufactory, operated by Sinopec Yangzi Petrochemical, is situated within 10 km of Nitroma’s selected site in Jiangsu. The Jiangsu province also hosts 20\% of China’s inland waterways (22,500 km) and 4,710 km of highways, creating reliable access to the neighbouring Shanghai Port and large markets in Shandong and Zhegiang. Moreover, Jiangsu operates a favourable policy towards foreign-owned and private businesses (XX). 

\subsection{Business model} 
\subsubsection{Business plan}
Nitroma will operate as a contract manufacturing organisation (CMO), developing a continuous nitration process to produce 4-aminobenzaldehyde (≥ 98\%), 4-aminobenzoic acid (≥ 98\%) and o-toluidine (≥ 95\%). The target markets for these products are the local dye, pharmaceutical and agrochemical industries in China.
\subsubsection{Industry landscape}
The Chinese markets for Nitroma’s three products are highly fragmented, with the average market share for 4-aminobenzaldehyde, 4-aminobenzoic acid and o-toluidine summarised in Table XX. The industry structure can be defined as monopolistic competition where Nitroma reserves pricing power. In recent times, growing concerns around safety have led Chinese authorities to introduce upgraded safety legislation and operating requirements for chemical sites (XX). In 2019, Jiangsu shut down 9 chemical parks (equivalent to 244 chemical plants) due to safety risks (XX). Incumbents and newcomers in the Chinese chemicals market will have to adapt fast to survive the challenging phase of increased regulatory control. 
\subsubsection{Value proposition}
Nitroma seeks the reward of fulfilling unmet demand as many of its competitors are removed from the market. Nitroma’s inherently safe design means it is not threatened by closures, making it a more reliable supplier for buyers to enter long-term purchasing contracts with. The unique continuous nature of Nitroma’s nitration process also means improved product quality – an important characteristic for Nitroma’s pharmaceutical customer segment. Moreover, Nitroma’s built-in production flexibility gives buyers the option to renegotiate contracts as their needs evolve. Finally, Nitroma offers convenience to buyers due its diversity of aromatic amine products, making it an attractive “one-stop shop”.
\subsubsection{Economic model: cost leadership}
Nitroma is positioned to leverage recent tax subsidisation policies set by the Chinese Ministry of Commerce (XX) related to safe process designs [XX]. Nitroma aims to follow the example of its industry peers, such as Yangmei Chemical which received 62 subsidies worth USD 10.1 million for transitioning to energy-efficient equipment in 2020 [XX]. Moreover, employing a unique continuous process will allow Nitroma to cut operating costs by improving its hourly production rate and resource efficiency, whilst reducing operating expenditure. As a result, Nitroma can set a below-market price point for its product line and still generate a healthy profit margin. This competitive advantage will allow Nitroma to steal market share from current incumbents. 
\subsubsection{Mode of Operation – WFOE}
Nitroma will establish itself as a wholly foreign-owned enterprise (WFOE) in China. Being a high-tech company with a physical footprint, operating as a WFOE will allow Nitroma to register a legal presence in China and so, own real estate and intellectual property under its own name. Another important aspect of operating as a WFOE is that Nitroma will have independent management of its processes, making it easier to negotiate contracts with buyers under its CMO-styled operations. The risks associated with this decision include cultural obstacles, language barriers and lengthy wait times for governmental approvals.

\subsection{Production capacity}
Nitroma’s annual production and market share targets are summarised in Table XX. These numbers are based on the local product demand and the average market share of competitors, detailed in Table XX. The targets are set for the first 5 years of Nitroma’s operations, with the possibility of a future plant expansion currently under investigation. Nitroma is confident its business strategy, outlined in Section XX, will allow it to achieve these targets. 

\subsection{Market analysis}

\subsubsection{Feedstock Supply}
Toluene is used in construction, explosions and agrochemicals which drives the domestic demand for it in China []. The price of toluene is susceptible to fluctuating prices in crude oil which leads to fluctuating global supply and demand []. China had plant expansions in 2019 [] therefore domestic production of toluene is expected to increase and provide reliable supply[]. Toluene will be sourced from Sinopec Yangzi Petrochemical in Jiangsu, China[].

\subsection{Economic analysis}