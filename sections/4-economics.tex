% !TeX root = ../main.tex
\section{Business assessment}
\label{sec:economics}
\subsection{Selection of location} 
AHP and TOPSIS methodologies were used to determine a suitable country for Nitroma’s chemical plant. The key factors taken into consideration are summarised in Table XX in Appdendix XX. China was identified as the optimal country for Nitroma's plant, owing to its strong local supply chain and low business operating costs. A sensitivity analysis was performed on the decision criteria, but none of the scenarios downgraded China. To identify an optimal city within China, the spread of toluene suppliers, access to distribution channels, local market demand and business policies across the provinces of China were studied. The Nanjing Jiangbei New Material Science Park located in Nanjing, Jiangsu was selected. China’s largest toluene manufactory, operated by Sinopec Yangzi Petrochemical, is situated within 10 km of Nitroma’s selected site in Jiangsu (XX). The Jiangsu province also hosts 20\% of China’s inland waterways and 4,710 km of highways, creating reliable access to the neighbouring Shanghai Port and markets in Shandong and Zhegiang (XX). Moreover, Jiangsu operates a favourable policy towards foreign-owned and private businesses (XX). 

\subsection{Business model} 
\subsubsection{Business plan}
\begin{wraptable}{r}{0.4\linewidth}
    \vspace{-\intextsep}
    \centering\footnotesize
    \caption{Production capacity}
    \label{tab:production-capacity}
    \begin{tabular}{lcc} 
    \toprule
    Product & \splitcell{Annual\\ production\\ (\si{\tonne\per\year})} & \splitcell{Market\\ share (\%)} \\ \midrule
    4-aminobenzaldehyde & 0-500 & 1.50 \\ 
    4-aminobenzoic acid & 0-100 & 3.00 \\ 
    o-toluidine & 0-650 & 0.10 \\ 
    \bottomrule
    \end{tabular}
\end{wraptable}
Nitroma will operate as a contract manufacturing organisation (CMO) producing 4-aminobenzaldehyde (≥ 98\%), 4-aminobenzoic acid (≥ 98\%) and o-toluidine (≥ 95\%). The target markets for these products are the local dye, pharmaceutical and agrochemical industries. Nitroma’s  production and market share targets are summarised in \cref{tab:production-capacity}, based on  product demand and the average market shares in China (see \cref{tab:product}). The targets are set for the first 5 years of Nitroma’s operations, with the possibility of future expansion under investigation.
\subsubsection{Industry landscape}
The Chinese markets for Nitroma’s three products are highly fragmented, with the average market share for 4-aminobenzaldehyde, 4-aminobenzoic acid and o-toluidine summarised in Table 1. The industry structure can be defined as monopolistic competition where Nitroma reserves pricing power. In recent times, growing concerns around safety have led Chinese authorities to introduce upgraded safety legislation, with Jiangsu shutting down 244 chemical plants in 2019. Incumbents and newcomers in the Chinese chemicals market will have to adapt fast to survive increased regulatory control.
\subsubsection{Value proposition}
Nitroma seeks the reward of fulfilling unmet demand as many of its competitors are removed from the market. Nitroma’s inherently safe design means it is not threatened by closures, making it a more reliable supplier for buyers to enter long-term purchasing contracts with. The unique continuous nature of Nitroma’s nitration process also means improved product quality – an important characteristic for Nitroma’s pharmaceutical customer segment. Moreover, Nitroma’s built-in production flexibility gives buyers the option to renegotiate contracts as their needs evolve. Finally, Nitroma offers convenience to buyers due its diversity of aromatic amine products, making it an attractive “one-stop shop”.
\subsubsection{Economic model: cost leadership}
Nitroma is positioned to leverage recent tax subsidisation policies set by the Chinese Ministry of Commerce related to safe process designs [XX]. Nitroma aims to follow the example of its peers, such as Yangmei Chemical which received 62 subsidies worth USD 10.1 million for transitioning to energy-efficient equipment in 2020 [XX]. Moreover, employing a continuous process will allow Nitroma to cut operating costs by improving its hourly production rate and resource efficiency. As a result, Nitroma can set a below-market price point for its product line. This competitive advantage will allow Nitroma to acquire market share from industry incumbents. 
\subsubsection{Mode of operation: WFOE}
Nitroma will establish itself as a wholly foreign-owned enterprise (WFOE) in China. Being a high-tech company with a physical footprint, operating as a WFOE will allow Nitroma to register a legal presence in China and so, own real estate and intellectual property under its own name. Another important aspect of operating as a WFOE is that Nitroma will have independent management of its processes, making it easier to negotiate contracts with buyers under its CMO-styled operations.

%\subsection{Production capacity}
%\begin{wraptable}{r}{0.4\linewidth}
    %\vspace{-\intextsep}
    %\centering\footnotesize
    %\caption{Production capacity}
    %\label{tab:production-capacity}
    %\begin{tabular}{lcc} 
    %\toprule
    %Product & \splitcell{Annual\\ production\\ %(\si{\tonne\per\year})} & \splitcell{Market\\ share %(\%)} \\ \midrule
    %4-aminobenzaldehyde & 0-500 & 1.50 \\ 
    %4-aminobenzoic acid & 0-100 & 3.00 \\ 
    %o-toluidine & 0-650 & 0.10 \\ 
    %\bottomrule
    %\end{tabular}
%\end{wraptable}
%Nitroma’s annual production and market share targets are summarised in \cref{tab:production-capacity}. These numbers are based on the local product demand and average market share of competitors, detailed in \cref{tab:product}. The targets are set for the first 5 years of Nitroma’s operations, with the possibility of a future expansion under investigation. 

\subsection{Market analysis}
\subsubsection{Products: 4-ABH, 4-ABA and O-TOL}
4-ABH is used as ... 4-ABA is used as a precursor in the manufacture of drugs, with up to 1.5\% of all drugs containing the chemical (XX). As public health reforms expand in China, the country's share of the global pharmaceutical market is expected to grow from 11.54\% in 2014 to 30\% in 2030, only second to the United States (XX). O-TOL is used as a precursor in the manufacture of agrochemicals, specifically the herbicides metolachlor and acetochlor. Asia-Pacific, with China as its largest market, is the fastest growing herbicides market globally with an expected CAGR of 6.5\% between 2018 and 2025 (XX). Nitroma aims to capture the rapid growth of these industries. 
\subsubsection{Raw material: Toluene}
Toluene is used in construction, explosions and agrochemicals which drives the domestic demand for it in China \cite{reportbuyer_global_nodate} . The price of toluene is susceptible to fluctuating prices in crude oil which leads to fluctuating global supply and demand \cite{noauthor_toluene_nodate-1}. China had plant expansions in 2019 \cite{zhang_china_nodate} therefore domestic production of toluene is expected to increase and provide reliable supply \cite{noauthor_toluene_nodate}. Toluene will be sourced from Sinopec Yangzi Petrochemical in Jiangsu, China[].

\subsection{Economic analysis}