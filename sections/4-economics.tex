% !TeX root = ../main.tex
\section{Business Assessment}
\label{sec:economics}
\subsection{Selection of Location} 
AHP and TOPSIS methodologies were used to determine a suitable country for Nitroma’s chemical plant. The key factors taken into consideration are summarised in \cref{tab:location-selection}. China was identified as the optimal country for Nitroma's plant, owing to its strong local supply chain and low business operating costs. To identify an optimal city within China, the spread of toluene suppliers, access to distribution channels, local market demand and business policies across the provinces of China were studied. The Nanjing Jiangbei New Material Science Park located in Nanjing, Jiangsu was selected. China’s largest toluene manufacturer, operated by Sinopec Yangzi Petrochemical, is situated within \SI{10}{\km} of Nitroma’s selected site in Jiangsu (XX). The Jiangsu province also hosts \SI{20}{\percent} of China’s inland waterways and \SI{4710}{\km} of highways, creating reliable access to the neighbouring Shanghai Port and large markets in Shandong and Zhejiang (XX). Moreover, Jiangsu operates a favourable policy towards foreign-owned and private businesses (XX). 

% A sensitivity analysis was performed on the decision criteria, but none of the scenarios downgraded China. 

\subsection{Business Model} 
\subsubsection{Business Plan}
% \begin{wraptable}{r}{0.4\linewidth}
%     \vspace{-\intextsep}
%     \centering\footnotesize
%     \caption{Production targets}
%     \label{tab:production-targets}
%     \begin{tabular}{lcc} 
%     \toprule
%     Product & \splitcell{Production\\ (\si{\tonne\per\year})} & \splitcell{Market\\ share (\%)} \\ \midrule
%     4-aminobenzaldehyde & 0-500 & 1.50 \\ 
%     4-aminobenzoic acid & 0-100 & 3.00 \\ 
%     o-toluidine & 0-650 & 0.10 \\ 
%     \bottomrule
%     \end{tabular}
% \end{wraptable}

Nitroma will operate as a contract manufacturing organisation (CMO) producing 4-ABH (\SI{>= 98}{\percent}), 4-ABA (\SI{>= 98}{\percent}) and o-TOL (\SI{>= 95}{\percent}). The target markets for these products are the local dye, pharmaceutical and agrochemical industries in China. Nitroma’s production and market share targets are summarised in \cref{tab:brief} in \cref{app:brief}, based on product demand and the average market shares of competitors in China (see \cref{tab:product}). The targets are set for the first 5 years of Nitroma’s operations, with the proposition of expansion under review.

\nomenclature[A]{CMO}{Contract manufacturing organisation}

\subsubsection{Industry Landscape}
The Chinese markets for Nitroma’s three products are highly fragmented. The industry structure can be defined as monopolistic competition where Nitroma reserves pricing power. In recent times, growing concerns around safety have led Chinese authorities to introduce upgraded safety legislation, with Jiangsu shutting down 244 chemical plants in 2019 []. Incumbents and newcomers in the Chinese chemicals market will have to adapt fast to survive increased regulatory control.

%, with the average market share for 4-ABH, 4-ABA and o-TOL summarised in

\subsubsection{Value Proposition}
Nitroma seeks the reward of fulfilling unmet demand as many of its competitors are removed from the market. Nitroma’s inherently safe design means it is not threatened by closures, making it a more reliable supplier for buyers to enter long-term purchasing contracts with. The unique continuous nature of Nitroma’s nitration process also means improved product quality – an important characteristic for Nitroma’s pharmaceutical customer segment. Moreover, Nitroma’s built-in production flexibility gives buyers the option to renegotiate contracts as their needs evolve. %Finally, Nitroma offers convenience to buyers due its diversity of aromatic amine products, making it an attractive “one-stop shop”.

\subsubsection{Economic Model: Cost Leadership}
Nitroma is positioned to leverage recent tax subsidisation policies set by the Chinese Ministry of Commerce related to safe process design [XX]. Nitroma aims to follow the example of its peers, such as Yangmei Chemical which received 62 subsidies worth USD 10.1 million for transitioning to energy-efficient equipment in 2020 [XX]. Moreover, employing a continuous process will allow Nitroma to cut operating costs by improving its hourly production rate and resource efficiency. As a result, Nitroma can set a below-market price point for its product line. This competitive advantage will allow Nitroma to steal market share from industry incumbents. 

\subsubsection{Mode of Operation: WFOE}
Nitroma will establish itself as a wholly foreign-owned enterprise (WFOE) in China. Being a high-tech company with a physical footprint, operating as a WFOE will allow Nitroma to register a legal presence in China and so, own real estate and intellectual property under its own name [XX]. Another important aspect of operating as a WFOE is that Nitroma will have independent management of its processes, making it easier to negotiate contracts with buyers under its CMO-styled operations.

\nomenclature[A]{WFOE}{Wholly Foreign-Owned Enterprise}

\subsubsection{Security of Supply}
Asia-Pacific is the world's largest toluene producing region (XX), with China responsible for \SI{56}{\percent} of the region’s output, equivalent to 4.3 million tonnes/yr (XX). As Nitroma prepares to enter negotiations with key suppliers in Jiangsu and neighbouring provinces, the company aims to take advantage of the drop in toluene prices following the COVID-19 pandemic. Continued toluene plant expansions in China give Nitroma confidence in a consistent raw material supply (XX).

\subsection{Market Analysis: 4-ABH, 4-ABA and o-TOL}
4-ABH is used as a precursor in the manufacture of dyes. Currently the second largest dye market in the world, China is also the second fastest-growing dye market with a CAGR of \SI{5.04}{\percent} (2020-2024) due to its expanding textiles demand (XX). 4-ABA is used as an intermediate in the manufacture of drugs, with up to \SI{1.5}{\percent} of all drugs being a derivative (XX). As public health reforms are initiated in China, the country’s share of the global pharmaceutical market is expected to grow from \SI{11.54}{\percent} in 2014 to \SI{30}{\percent} in 2030, only second to the United States (XX). o-TOL is used as a precursor in the manufacture of agrochemicals, specifically the herbicides metolachlor and acetochlor. Asia-Pacific, with China as its largest market, is the fastest growing herbicides market globally with an expected CAGR of \SI{6.5}{\percent} (2018-2025) (XX). Nitroma, situated in the centre of China’s chemicals industry, aims to capture the rapid growth of these industries.  

\subsection{Economic Analysis}
An economic potential analysis was conducted to determine the financial feasibility of Nitroma’s operations. The annual profit was determined by deducting the cost of feed from the revenue generated by Nitroma's product line. The annual profit ranged from \$5.2m to \$12.0m, as Nitroma's operating time was varied from \SI{100}{\percent} under the ABA scenario to \SI{100}{\percent} under the ABH scenario. Key operating costs like utilities and revenue from by-products such as PNT were not considered at this stage. The capital cost of the plant was estimated at \$46.9m using the Bridgwater method. Using the aforementioned annual profit range, this led to a breakeven period of 3.5 to 8 years. All detailed financial calculations are found in \cref{app:economics}.