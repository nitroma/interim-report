% !TeX root = ../main.tex
\section{Environmental, Health and Safety (EHS) Considerations}
\label{sec:ehs}
\subsection{Health and Safety}

Inherent safety, hazard identification and risk analysis were fundamental elements in the design of the process plant. 

Several actions were taken to achieve an inherently safer design. The continuous nature of the process as adopted presents a safer option than a batch process. Toluene is the initial feedstock substance for all three products manufactured, minimising the hazardous inventory on-site. The substitutions made include the use of H-mordenite as the nitration catalyst, favouring this zeolite over sulfuric acid. Where possible, hazardous processes have been isolated from each other, as viewed seen in the P&ID and simplifications have been ensured, only necessary equipment included in the design. 

Dow's Fire and Explosion Index was utilised as an initial assessment to identify the specific process unit that holds the most potential to contribute to a fire, blast or explosion, and subsequently quantify the damage it would inflect should the incident occur. The specific hazards of each chemical component have considered, with the presence of the NFPA 704 values, the flash point value, the auto-ignition temperature and hazard codes. The chemical and biological health hazards are present in the table to indicate health risk faced upon exposure. A qualitative risk assessment was also conducted. % finish sentence and show in the appendix, and mention that controls  have been put in place. %  The hazard identification and risk analysis allow the implementation of appropriate controls to mitigate potential process hazards. 


\subsection{Environmental Impact}
