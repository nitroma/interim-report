% !TeX root = ../main.tex
\section{Environmental, Health and Safety (EHS) Considerations}
\label{sec:ehs}

\nomenclature[A]{EHS}{Environmental, health and safety}

Nitroma demonstrates the importance of EHS considerations are by complying to all relevant safety and environmental laws and regulations, as listed in Appendix \ref{app:EHS}. 

\subsection{Health and Safety}
Inherent safety was a fundamental element of the design stage. The continuous nature of the process presents a safer alternative to batch operation, as it requires a smaller inventory and reduces threat of thermal runaway caused by hot-spot formation. Furthermore, a continuous process will allow adaptable automatic control, eliminating the need for manual handling and thus reducing risks associated with human error \cite{mannan_lees_2012}. A key substitution was the choice of nitration catalyst, with H-mordenite favoured over concentrated sulphuric acid which is highly corrosive and reactive. Moderations were made in terms of plant layout, isolating the hazardous process units as much as reasonable, preventing sequential events should a fire or explosion occur. To simplify the overall design of the plant a minimal number of reaction and separation units were chosen, lowering the complexity of the plant whilst enabling specified process targets to be met. 

%Inherent safety was a fundamental element of the design stage. The hazardous inventory has been reduced by using toluene as the initial feedstock material for all final products. The continuous nature of the process presents a safer alternative to a batch operation, as it requires a smaller inventory and reduces the threat of thermal runaway due to hot-spot formation. Furthermore, a continuous process will allow adaptable automatic control, eliminating the need for manual handling, thus thus reduces risks associated with human error \cite{mannan_lees_2012}. A key substitution was the choice the nitration catalyst, with H-mordenite favoured over concentrated sulphuric acid which is highly corrosive and reactive. Moderations were made in terms of plant layout, isolating the hazardous process units as much as possible, preventing sequential events should a fire or explosion occur. To simplify the overall design of the plant a minimal number of reaction and separation units were chosen, lowering the complexity of the plant whilst enabling specified process targets to be met. 

The specific hazards of each chemical component have been considered, with the NFPA 704 values, flash point value, auto-ignition temperature, vapour pressure and Globally Harmonized System Hazards found in\Cref{tab:chemID}. All substances have a relatively low reactivity rating. There are two oxidisers present (nitric acid and oxygen) and the most flammable substance identified was hydrogen gas.  

Fires, explosions and the release of toxic substances are the typical consequences of the major process hazards which can occur in a chemical plant \cite{mannan_lees_2012}. For each of the main hazards identified that may result in the occurrence of these events, the risk was estimated in a qualitative approach using the severity-likelihood risk matrix, found in \Cref{tab:risk-matrix}. For a quantitative hazard assessment, Dow's Fire and Explosion Index was utilised to identify which specific process units are most liable to fires, blasts and explosions occurring, sizing the amount of damage it would inflect should the incident occur (\Cref{app:EHS}). The hazard degree ranged from light to moderate, with the toluene nitration in R101 owning the highest fire and explosion index. This essential identification and analysis permit the selection of appropriate actions and controls to mitigate potential process hazards. This includes integrating appropriate temperature and pressure monitors into reactors, the fitting of cooling jackets and firewalls, and the selection of material best suited to prevent loss of containment.


\subsection{Environmental Impact}

The modular nature of the process plant indicates waste streams will be identified at various units across the plant. Following guidance from the Green Engineering Principles and the waste hierarchy system, efforts have been made to prevent waste where possible. The traditional method of nitration involves the use of sulphuric acid, however disposal of the ‘spent’ acid is environmentally unfriendly and costly and thus a recyclable zeolite catalyst was chosen \cite{smith_superior_1996}.  Formic acid was chosen as a reducing agent for the hydrogenation reactions over ammonium formate to prevent the emission of ammonia. Furthermore, energy recovery is being considered from the waste stream of nitration by utilising the energy for heat exchange from the nitric acid and water mixture. 
%The toluene for the nitration solvent will be re-used through a recovery system.

Despite efforts made, the presence of waste is inevitable and so a list of the best available techniques was devised (\cref{tab:treat}), the treatment ensuring compliance with regarding threshold limits (\cref{tab:waste}). 