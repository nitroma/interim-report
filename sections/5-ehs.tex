% !TeX root = ../main.tex
\section{Environmental, Health and Safety (EHS) Considerations}
\label{sec:ehs}
\subsection{Health and Safety}

Inherent safety, hazard identification and risk analysis were fundamental elements in the design of the process plant. 

Several actions were taken to achieve an inherently safer design. The continuous nature of the process as adopted presents a safer option than a batch process. Toluene is the initial feedstock for all three products manufactured, minimising the hazardous inventory on-site. The substitutions made include the use of H-mordenite as the nitration catalyst, favouring this zeolite over sulfuric acid. Where possible, hazardous processes have been isolated from each other, as viewed seen in the P&ID and simplifications have been ensured, only necessary equipment included in the design. 

The specific hazards of each chemical component have been considered, with the presence of the NFPA 704 values, the flash point value, the auto-ignition temperature and hazard codes. The chemical and biological health hazards are present in the table to indicate health risk faced upon exposure.  

Fires, explosions and the release of toxic substances are the typical consequences of the major process hazards which can occur in a chemical plant. For each of the main hazards identified in Nitroma's process plant that may result in the occurrence of these events, the risk was estimated in a qualitative approach using the severity-likelihood risk matrix. For a quantitative assessment, Dow's Fire and Explosion Index was utilised, identifying which specific process units are most liable to the occurrence of a fire, blast or explosion and subsequently size the amount of damage it would inflect should the incident occur. 

The hazard identification and risk analysis allow the implementation of appropriate controls to mitigate potential process hazards. 


\subsection{Environmental Impact}

The modular nature of the process plant indicates waste streams will need to be defined for each of the different reaction processes that occur. To comply with environmental regulations, (ensuring the emissions of each component will fall below their respective strict threshold limit), various treatment and solvent recovery strategies have been considered. 