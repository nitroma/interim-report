% !TeX root = ../main.tex
\section{Environmental, Health and Safety (EHS) Considerations}
\label{sec:ehs}
\subsection{Health and Safety}

Inherent safety, hazard identification and risk analysis were fundamental elements in the design of the process plant. 

Each of the inherent safety principles (minimisation, substitution, moderation and simplification) were employed in the initial design stage of the process plant. The hazardous inventory has been reduced by utilising toluene as the initial feedstock material for all three of the products manufactured. The continuous nature of the process presents a safer option than the batch operation alternative, as it requires a smaller inventory of substances, eliminates the need for manual handling and reduces threat of thermal runaway due to hot-spot formation. A key substitution was the choice of catalyst for the nitration, with the H-mordenite zeolite favoured over the more classic option, concentrated sulphuric acid which is highly corrosive and extremely reactive. Moderations were made in terms of plant layout, isolating the hazardous process units as much as possible, preventing sequential events should a fire or explosion with one process unit. The efforts were made to simplify the overall design of the plant, minimising the number of reaction and separation units present, lowering the complexity of the plant whilst enabling specified process targets to be met. 

The specific hazards of each chemical component have been considered, with the presence of the NFPA 704 values, the flash point value, the auto-ignition temperature, vapour pressure and hazard codes respectively. The chemical and biological health hazards are present in the table to indicate health risk faced upon exposure.  

Fires, explosions and the release of toxic substances are the typical consequences of the major process hazards which can occur in a chemical plant \cite{mannan_lees_2012}. For each of the main hazards identified in Nitroma's process plant that may result in the occurrence of these events, the risk was estimated in a qualitative approach using the severity-likelihood risk matrix. For a quantitative hazard assessment, Dow's Fire and Explosion Index was utilised, identifying which specific process units are most liable to the occurrence of a fire, blast or explosion and subsequently size the amount of damage it would inflect should the incident occur. 

%The hazard identification and risk analysis allow the implementation of appropriate controls to mitigate potential process hazards. 


\subsection{Environmental Impact}

The modular nature of the process plant indicates waste streams will be identified at various units across the plant. Following guidance from the 12 Principles of Green Engineering and the waste hierarchy system, efforts have been made to prevent waste where possible (as opposed to treating it). Whilst the traditional method of nitration involves the mixture of nitric and sulphuric acid as reagents, disposal of ‘spent’ sulphuric acids can be environmentally unfriendly and costly. Therefore, alternative method using recyclable zeolite catalyst will be utilised \cite{smith_superior_1996}. The toluene for the nitration solvent will be re-used through a recovery system. Formic acid was chosen as a reducing agent for the hydrogenation reactions over ammonium formate to prevent the emission of  ammonia. Energy recovery is being considered from the waste stream of nitration by utilising the energy for heat exchange from the nitric and water mixture. However, despite efforts made, the presence of waste substances is inevitable. Where in cases disposal is unavoidable, the waste will be treated to ensure compliance with environmental regulations regarding threshold limits. The national wastewater treatment works regulatory body states that the maximum threshold limits for the biological oxygen demand and chemical oxygen demand are \SIlist{50;250}{\mg\of{\ch{O2}}\per\l} respectively, whilst the maximum concentration limit of total nitrogen is \SI{15}{\mg\of{\ch{N}}\per\l} \cite{noauthor_waste_nodate}.